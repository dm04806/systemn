\documentclass[a4paper,5pt,onecolumn]{article}

% truc de base

% Permission is granted to copy, distribute and/or modify this
% document under the terms of the GNU Free Documentation License,
% Version 1.2 or any later version published by the Free Software
% Foundation; with no Invariant Sections, no Front-Cover Texts and
% no Back-Cover Texts.  A copy of the license is included in the
% section entitled "GNU Free Documentation License".
%
% Copyright (C) 2008 Nicolas Marti

\usepackage[latin1]{inputenc}
\usepackage[english]{babel}
\usepackage[T1]{fontenc}
%\usepackage[width=20cm,height=29cm]{geometry}
\usepackage[width=18cm,height=25cm]{geometry}
%\usepackage{verbatim}
% pour les graphique ext
\usepackage{graphicx}

%pour que l'env math marche enfin !!!! X(
\usepackage{amsmath}
\usepackage{amsfonts}
\usepackage{color}
\usepackage{proof}

\usepackage{hyperref}
\hypersetup{colorlinks, 
           citecolor=black,
           filecolor=black,
           linkcolor=black,
           urlcolor=black,
           bookmarksopen=true,
           pdftex}

\hfuzz = .6pt % avoid black boxes

\usepackage{url}

\newcommand{\bnfeq}{::=}
\newcommand{\matchwith}[2]{{\tt match \ } #1 {\tt \ with\ } #2}
\newcommand{\consof}[2]{{\tt cons \ } #1 {\tt \ of\ } #2}
\newcommand{\type}{{\tt type}}
\newcommand{\myinfer}[3]{\infer[\texttt{#1}]{#2}{#3}}
\newcommand{\betarel}[1]{\rightarrow^{#1}_\beta}
\newcommand{\alpharel}[1]{\rightarrow^{#1}_\alpha}

\def\us{\char`\_}


%pour l'index
\usepackage{makeidx}

\title{Mymms Programming Language Tutorial}

\author{Nicolas MARTI}


\begin{document}

\maketitle

\abstract{This tutorial require good knowledge about functional
  programing language, and logic. Experience with a proof assistant
  (especially Coq) is a plus, but is not required.}

\section{Introduction}

Mymms includes:
\begin{itemize}
\item a programming language based on the Calculus of Inductive
  Construction, extensible with primitives (types, functions, ...),
  and oracles (functions which goal is to produce a term of a given
  type, used by the typechecking algorithm for inference and
  non-trivial unification over types), in ocaml
\item an interactive proof mode (a la Coq)
\item a jit compiler (Partially implemented)
\item a python virtual machine that can use all the previous features
  (Partially implemented)
\end{itemize}

Please consult the README file in the root of the source to properly
compile mymms and set the require environment variables.

\section{The programming language}

In this part of the tutorial we will first focus on the programming
language. We will first introduce the basics of the language through a
set a illustrating examples. Then we will review more formally all the
language constructions. Eventually, we will resume all the presented
language features into more advanced examples.

Before starting, one remark: whereas most of functional programming
language, like ocaml or haskell, include separates languages for terms
and types, the calculus of construction (and all its extensions, such
as the calculus of inductive construction, further refered as CiC)
merges both terms and types into a unique syntax. This is actually the
source of its power!

First thing first, let's start our serie of examples with the
definition of data-structures: inductive types (also known as GADTs).
The most simple one is peano implementation of natural numbers:

\begin{verbatim}

Inductive nat : Type :=
| O: nat
| S: nat -> nat.

\end{verbatim}

Save the following script into a file \texttt{tutorial1.m} and then
invoke mymms though \texttt{mymms -f tutorial.m}. If everything goes
well mymms should output:

\begin{verbatim}

------------- Definition -------------
O Defined
S Defined
nat Defined
natinduc Defined
--------------------------------------

\end{verbatim}

What happened? We created an inductive type named \texttt{nat} with
two constructor: \texttt{O} and \texttt{S}. More intuitively we
declared the type \texttt{nat} (itself of type \texttt{Type}), a set
of the term2 of type nat that can be build through the two
constructors (More formally the the kleene closure\footnote{so an
  infinite set of finite terms} of the constructors). This set/type
contains \texttt{0}, (\texttt{S} \texttt{O}), (\texttt{S} (\texttt{S}
\texttt{0})), ...

What is this \texttt{Type}??? Remember that in the CiC, types and
terms are merges into one language. \texttt{Type} is somehow the root
type. It is the type of the sets and of the logical propositions.

Mymms defined \texttt{O}, \texttt{S}, \texttt{nat},
\texttt{natinduc}. This last one is the induction principle of the
type \texttt{nat}: a basic, but useful lemma that can be used to prove
lemmas over the naturals.

We can ask the types of terms to mymms through the command
\texttt{Check}. For instance:

\begin{verbatim}

Check natinduc.

\end{verbatim}

will ouput

\begin{verbatim}

natinduc: V (Pred: nat -> Type). (Pred O) -> (V (Hyp: nat). (Pred Hyp) -> Pred (S Hyp)) -> V (x: nat). Pred x

\end{verbatim}

This term is a lemma. Let us explains this type. In second ordered
lambda calculus, a lambda abstraction can be written \texttt{$\lambda$
  (x: A). t} (in mymms syntax, \texttt{\ (x: A). t}), with its type
being \texttt{A -> B} (when \texttt{t} is of type B). Somehow
\texttt{->} is also a quantification, but where the variable is
omitted because it will not appears in B (as it is second
order). However it can be the case in CiC, which explicitely
quantifies the type on the l.h.s of an arrow (formally
\texttt{$\forall$ (a: A). B} (in mymms syntax, \texttt{V (x: A). t}),
however when the variable does not appears in the r.h.s the arrow
notation can be used).

Reading the \texttt{natinduc} lemma informally: 
\begin{itemize}
\item \texttt{V (Pred: nat -> Type)}: given a predicate \texttt{Pred}
  on elements of \texttt{nat}, ...
\item \texttt{(Pred O)} : given a proof that the predicate is valid for
  the element \texttt{O} of \texttt{nat}, ...
\item \texttt{(V (Hyp: nat). (Pred Hyp) -> Pred (S Hyp))}: given a
  proof that for any nat (\texttt{Hyp}), such that \texttt{Pred} is
  valid, then \texttt{Pred} is also valid for the next nat, ...
\item \texttt{V (x: nat). Pred x}: then the predicate \texttt{Pred} is
  valid for all the element in the set \texttt{nat}
\end{itemize}

This is a well-known induction for the natural, and mymms just give it
for free with the definition of the naturals! Yet, nothing fancy here
... though always having an inductive principle for every defined
inductive types is really convenient. Unfortunately, really difficult
lemmas are never solved directly with the induction principles.

So now that we defined the natural numbers, we should define some
functions on them. The first that comes in mind is the addition! In
mymms, this could be written as:

\begin{verbatim}

Recursive natplus (x y: nat) [0] : nat :=
  match x with
    | O ==> y
    | S x ==> S (natplus x y).

------------- Definition -------------
natplus Defined
--------------------------------------

\end{verbatim}

\texttt{Recursive} allows to define recursive functions. Here we
defined \texttt{natplus}, which takes two arguments (\texttt{x} and
\texttt{y}) of type \texttt{nat}, and returns a term of type
\texttt{nat}. You may be surprise to find the notation
\texttt{[0]}. This number is an index of arguments (here \texttt{x}),
which indicates to the function that its first arguments decrease in
size at each recursive invocation of \texttt{natplus}. This helps the
typechecker to verify that the recursive function will always be
terminating. Why is that? Because the CiC is strongly normalizing (all
computation finishes). This is a condition sine qua non for
lambda-calculus that are intend to represent proof. Intuitively,
without strong normalization, a lambda calculus could have a term (==
proof) of type \texttt{A -> B}, for any \texttt{A} and \texttt{B}
(imagine false -> true), which is obviously false. The proof (here in
ocaml):

\begin{verbatim}

let rec f x = f x

\end{verbatim}

We can further implement the multiplication and check its type (here
we removed all type annotation, mymms will not care):

\begin{verbatim}

Recursive natmul x y [0] :=
  match x with
    | O ==> O
    | S x ==> natplus y (natmul x y).

------------- Definition -------------
natmul Defined
--------------------------------------

Check natmul.

natmul: nat -> nat -> nat

\end{verbatim}

Ok everything is fine. Now let's implement something else:
lists. Lists are parameterized by the type of their element. In mymms:

\begin{verbatim}

Inductive list (A: Type) : Type :=
| nil: list A
| cons: A -> list A -> list A.

\end{verbatim}

When we want the inductive type to be parameterized (here by a type
\texttt{A}), we simply put arguments after the inductive type
name. This quantification will span over all the constructor (hence
\texttt{nil} and \texttt{cons} are in the scope of \texttt{A}).

Now we can build a list of naturals, for instance:

\begin{verbatim}

Check (cons nat O (cons nat O (nil nat))).

cons nat O (cons nat O (nil nat)): list nat

\end{verbatim}

However, it is cumbersome to always repeat \texttt{nat}. Indeed as it
is the type of the second argument of cons, mymms should be able to
infere it. We can verify it through using the special term \texttt{\_}
(which stand for a fresh variable that mymms will implement and infere
during typechecking).

\begin{verbatim}

Check (cons _ O (cons _ O (nil _))).

cons nat O (cons nat O (nil nat)): list nat

\end{verbatim}

Fine, but it would be better to avoid to put one \texttt{\_} after
both \texttt{nil} and \texttt{cons} each time we use them. We can tell
that the first argument of both constructors is implicite through the
following command:

\begin{verbatim}

Implicite cons [1].
Implicite nil [1].

\end{verbatim}

Thus we can create the same previous list as:

\begin{verbatim}

Check (cons O (cons O nil)).

cons O (cons O (nil)): list nat

\end{verbatim}

But what if I want an empty list of nat? The first arguments can only
be infered if I give it some type. However, standing alone mymms will
give us an error. For instance:

\begin{verbatim}

Check nil.

termchecking error (incomplete inference)!

unresolved  (?T: Type) in  (nil ?T: list ?T)

Typecheck: Error!!

\end{verbatim}

Mymms told us that the term does not type because its implicite
argument cannot be infered. To remove the implicite arguments from a
defined term, we put the \texttt{@} prefix.

\begin{verbatim}

Check @nil.
nil: V (A: Type). list A

\end{verbatim}

Let us write well-known functions for list:

\begin{verbatim}


Recursive map A B (f: A -> B) (l: list A) [3] :=
   match l with
     | nil ==> nil
     | cons hd tl ==> cons (f hd) (map _ _ f tl).
Implicite map [2].

Recursive foldl A B (f: A -> B -> A) a l [4] :=
   match l with
     | nil ==> a
     | cons hd tl ==>
       	     foldl _ _ f (f a hd) tl .
Implicite foldl [2].

Recursive foldr A B (f: B -> A -> A) a l [4] :=
   match l with
     | nil ==> a
     | cons hd tl ==>
       	     f hd (foldr _ _ f a tl).
Implicite foldr [2].

Recursive intercalate A a (l: list A) [2] :=
  match l with
    | cons hd1 (cons hd2 tl) ==> cons hd1 (cons a (intercalate _ a (cons hd2 tl)))
    | _ ==> l.
Implicite intercalate [1].

\end{verbatim}

Let us write a simple function to illustrate the termination notation
when it need to span on more than one arguments. We will write a
function taking two lists and returning their mixing.

\begin{verbatim}

Recursive mixlist A (l1 l2: list A) [1, 2] :=
  match l1 with
    | nil ==> l2
    | cons hd1 tl1 ==>
      	   cons hd1 (mixlist _ l2 tl1).

\end{verbatim}

Here the problem comes from the fact that we swap the lists on the
recursive call. Thus the second arguments does not necessarily
decrease in size. Yet, the sum of the two lists sizes decrease. To
indicate to mymms to look for the sum of several arguments, we just
give their indices list.

So far so good. But nothing that ocaml or haskell can't do. What about
more a complicated type. For instance a list which type contains its
size? We can implement them like:

\begin{verbatim}

Inductive listn (A: Type) : nat -> Type :=
| niln: listn A O
| consn: V (n: nat). A -> listn A n -> listn A (S n).

\end{verbatim}

But why is the type of \texttt{listn (A: Type)} is \texttt{nat ->
  Type}??? Actually the type of a set is always \texttt{Type} (look at
the final type of both constructor, it is indeed of type
\texttt{Type}). But somehow, we cannot put the size as an argument of
the inductive type, because it is not the same in both
constructor. Thus the \texttt{nat -> Type}. The \texttt{nat} is a
parameter which quantification is not in the scope of the
constructors.

The type checking for this kind of inductive type is much more
difficult (mainly due to unification). For instance, the append
function on \texttt{listn} will be typecheck, but the reverse list
will fail (to work fine the typechecker should be aware of the
commutativity of the addition over \texttt{nat}). 

If you want to test them, here they are (the first arguments of each
constructor of \texttt{listn} has been declared as implicite):

\begin{verbatim}

Recursive appendn A n1 n2 (l1: listn A n1) (l2: listn A n2) [3] : listn A (natplus n1 n2) :=
   match l1 with
     | niln ==> l2
     | consn n1 a l1 ==> consn _ a (appendn _ _ _ l1 l2).

Implicite appendn [3].

Recursive revn A n (l: listn A n) [2] : listn A n :=
   match l with
     | niln ==> niln
     | consn n e l ==> appendn (revn _ _ l) (consn _ e niln).

\end{verbatim}

In difficult situation (inference of unification), the kernel can ask
help from a set of oracles. These are decision procedure taking as
input a type, and giving for result a term supposely of the same
type. The kernel typechecks the result for verification. Currently the
type classes are implemented using an oracle (we will see it
later). It is scheduled to implement a decision procedure for equality
using lemmas of commutativity and associativity (and more generally
equality). Once performed, mymms should be able to typecheck correctly
\texttt{revn}.

But what is the real interest of the dependent types? Their basic role
is to cut invalid destructor. For instance, if I know that the size of
some list is greater than zero, then I should not consider the
destructor \texttt{niln}. Mymms, as most functional programming
language support the basic \texttt{let} construction, has in:

\begin{verbatim}

Definition fct x := let x := S x in natplus x x.

\end{verbatim}

The \texttt{Definition} command allow to define non-recursive
function. Another use of \texttt{let} is to destruct an inductive type
when their is only one possible destructor. For instance:

\begin{verbatim}

Definition fct A n (l: listn A (S n)) := let consn _ e l := l in e.

\end{verbatim}

We are destructing an inductive type with two constructors, yet we
know from the dependent type that the term can only be a
\texttt{consn}.

A really useful dependent type for proof assistant is the definition
of equally:

\begin{verbatim}

Inductive eq A (a: A) : A -> Type :=
| eqrefl: eq A a a.

\end{verbatim}

Basically, two terms are equal if they are ... the same term. It could
be surprising, but remember that it is only a definition. What is
important is what can we do with this definition. And actually, we can
do a lot! This definition of equality, is the univeral congruence for
typechecking over any term! If you don't believe me I will give you a
proof (and so you will see a little bit of the proof mode).

Here is the lemma I will prove: for all type \texttt{A}, and for two
of its elements \texttt{a} and \texttt{b}, for all predicat \texttt{P}
on \texttt{A}, if \texttt{P} is valid on \texttt{a}, then it is valid
on \texttt{b}. Or more formally:

\begin{verbatim}

Lemma congruence_eq: V (A: Type) (a b: A) (H: eq a b) (P: A -> Type). P a -> P b.

\end{verbatim}

We remind that the curry-howard isomorphism put in relation logical
proposition with types, and proof with term. Thus we need to define th
body of \texttt{congruence\_eq}. Rather than giving the function in a
whole, we use the proof mode, which can be seen as a way to
iteratively build a term of a given type.

Entering the previous lemma definition in mymms leads to the following state:

\begin{verbatim}

----------------------------
V (A: Type). V (a: A). V (b: A). (eq A a b) -> V (P: A -> Type). (P a) -> P b

\end{verbatim}


The line separates the variables that we can use (on the top, and for
now empty), and the type of the term we need to build. We will
introduce the quantified variables. Intuitively it means that we build
the term as a sequence of lambda abstractions quantifying a term that
still need to be given. However, as we are now under the lambdas, we
are in the scope of the variables.

\begin{verbatim}

intros.

Hyp : P a
P : A -> Type
H : eq A a b
b : A
a : A
A : Type
----------------------------
P b

\end{verbatim}

Thus we need to is to give a term of type \texttt{P b}. If we were
writing the lambda term be hand we should be able to destruct a term
through a \texttt{match .... with} instruction. To do the same in the
proof mode, the command is \texttt{destruct} followed by the
destructed term. The inductive type \texttt{eq} has only one
constructor \texttt{eqrefl}. By definition, the second and third
arguments are the same. What does happen when destructing? Both are
unified to the same variable. In mymms:

\begin{verbatim}

destruct H.

a1 : A
A1 : Type
Hyp : P b
P : A -> Type
H : eq A b b
b : A
a : A
A : Type
----------------------------
P b

\end{verbatim}

Now the goal is trivial, as the term \texttt{Hyp} has exactly the good
type, in mymms:

\begin{verbatim}

exact Hyp.

No more subgoals

\end{verbatim}

We can now exit the proof mode and let mymms verify the whole proof:

\begin{verbatim}

Qed.

proof complete.

\end{verbatim}

We finished our proof through \texttt{Qed.} and thus mymms does not
save our proof but an axiom (which does not compute). The proof mode
can also be used to build functions (like \texttt{map}, ...)  which
you may want to compute later. In this case you have to end the proof
with the command \texttt{Defined}.

This closed this first version of the tutorial. For further examples,
the reader should look at the standard library (repository
\texttt{stdlib}) and the tests for mymms (repository
\texttt{test/mymms})

[TODO: Class Type example, more advanced example (maybe the balanced
  tree)]

\subsection{GADTs}

TODO

\subsection{Functions}

TODO

\subsubsection{Non recursive functions}

TODO

\subsubsection{Recursive functions}

TODO

\subsection{Advanced Examples}

\subsection{Proof Mode}

\section{Commands}

\end{document}


